\documentclass[
  ngerman,
  DIV=14
]{scrartcl}
\usepackage{babel}
\usepackage{csquotes}

% typography
\usepackage{fontspec}
%\usepackage[utopia]{mathdesign}
\usepackage{newpxmath}
\setsansfont{Open Sans}[
  BoldFont={Open Sans Bold},
  ItalicFont={Open Sans Italic}]
\setmonofont[Scale=0.87]{Menlo}
\setmainfont{Palatino}
\linespread{1.15}
%\renewcommand\familydefault{\sfdefault}
\usepackage[factor=1000]{microtype}

% graphics, drawings, etc.
\usepackage{xcolor}
\usepackage{graphicx}
\usepackage{tikz}
\usetikzlibrary{shapes.geometric}
\usetikzlibrary{shapes.arrows}
\usetikzlibrary{positioning}

% highlighting, lists, code
\usepackage{listings}
\lstset{
  basicstyle=\ttfamily,
  %escapeinside=||,
  keywordstyle=\color{blue!50!black},
  stringstyle=\color{green!50!black}}

% math
\usepackage{amsmath}

% links
\usepackage[
  colorlinks,
  linkcolor={red!50!black},
  citecolor={blue!50!black},
  urlcolor={blue!80!black}
]{hyperref}

\subject{Visual Computing}
\title{Übungsblatt 2}
\subtitle{Lösung}
\author{Patrick Elsen, Dmytro Klepatskyi,\\Costa Weiland, Nana Atchoukeu Chris-Mozart}
\date{Wintersemester 2018-2019}
\publishers{Technische Universität Darmstadt}

\begin{document}
\maketitle

\subsection*{Aufgabe 1: Das Drei-Stufen-Modell}
\emph{Nennen Sie die drei Stufen der menschlichen Informationsverarbeitung. Falls die Stufe weiter in Untersysteme aufgeteilt wurde, nennen Sie auch diese.}
\par\medskip\noindent
Antwort.

\bigskip\noindent
\emph{Geben Sie zwei Beispiele für Alltagssituationen, in denen das 3-Stufenmodell angewendet wird. Unterteilen Sie die Alltagssituation dabei in die 3 Stufen und geben Sie für jede Stufe die jeweilige Aktion an.}
\par\medskip\noindent
Antwort.

\subsection*{Aufgabe 2: Wahrnehmung}
\emph{Was ist Vektion und wie kommt Sie zu Stande?}
\par\medskip\noindent
Antwort.

\bigskip\noindent
\emph{Nennen Sie den Unterschied zwischen Reiz und Wahrnehmung.}
\par\medskip\noindent
Antwort.

\bigskip\noindent
\emph{Woran liegt es, dass sich Wahrnehmungen des gleichen Reizes unterscheiden können? Nennen Sie 2 Gründe.}
\par\medskip\noindent
Antwort.

\subsection*{Aufgabe 3: Frühe Wahrnehmung}
\emph{Beschreiben Sie kurz in eigenen Worten, was man unter „früher Wahrnehmung“ versteht. Geben Sie außerdem 3 Beispiele für frühe Wahrnehmung an, die nicht in den Vorlesungsfolien genannt werden.}
\par\medskip\noindent
Antwort.

\subsection*{Aufgabe 4: Depth Cue Theorie}
\emph{Nennen Sie die 3 Kategorien von Depth Cues.}
\par\medskip\noindent
Antwort.

\bigskip\noindent
\emph{Geben Sie für jede Kategorie jeweils ein Beispiel an und erklären Sie dieses kurz.}
\par\medskip\noindent
Antwort.

\end{document}