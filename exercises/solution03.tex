\documentclass[
  ngerman,
  DIV=14
]{scrartcl}
\usepackage{babel}
\usepackage{csquotes}

% typography
\usepackage{fontspec}
%\usepackage[utopia]{mathdesign}
\usepackage{newpxmath}
\setsansfont{Open Sans}[
  BoldFont={Open Sans Bold},
  ItalicFont={Open Sans Italic}]
\setmonofont[Scale=0.87]{Menlo}
\setmainfont{Palatino}
\linespread{1.15}
%\renewcommand\familydefault{\sfdefault}
\usepackage[factor=1000]{microtype}

% graphics, drawings, etc.
\usepackage{xcolor}
\usepackage{graphicx}
\usepackage{tikz}
\usepackage{soul}
\usetikzlibrary{shapes.geometric}
\usetikzlibrary{shapes.arrows}
\usetikzlibrary{positioning}

% highlighting, lists, code
\usepackage{listings}
\lstset{
  basicstyle=\ttfamily,
  %escapeinside=||,
  keywordstyle=\color{blue!50!black},
  stringstyle=\color{green!50!black}}

% math
\usepackage{amsmath}
\usepackage{siunitx}

% links
\usepackage[
  colorlinks,
  linkcolor={red!50!black},
  citecolor={blue!50!black},
  urlcolor={blue!80!black}
]{hyperref}

\subject{Visual Computing}
\title{Übungsblatt 2}
\subtitle{Lösung}
\author{Patrick Elsen, Dmytro Klepatskyi,\\Costa Weiland, Nana Atchoukeu Chris-Mozart}
\date{Wintersemester 2018-2019}
\publishers{Technische Universität Darmstadt}

\begin{document}
\maketitle

\subsection*{Aufgabe 1: Das Drei-Stufen-Modell}
\emph{Nennen Sie die drei Stufen der menschlichen Informationsverarbeitung. Falls die Stufe weiter in Untersysteme aufgeteilt wurde, nennen Sie auch diese.}
\par\medskip\noindent
Die drei Stufen der menschlichen Informationsverarbeitung sind die Wahrnehmung von Eindrücken durch die Sinne (\emph{sensory perception}), die Entscheidungsfindung im Gehirn (\emph{cognition}), sowie die Reaktion des Körpers (\emph{motor response}). Die erste Stufe, die Wahrnehmung, kann man in drei unterschiedliche Systeme einteilen, die alle parallel laufen. Dazu gehört das visuelle System (Sehen), das akustische System (Hören) und das haptische System (Fühlen). Ebenso kann man die Reaktion in mehrere Untersysteme unterteilen, unter anderem in das stimmliche System (Sprechen) sowie das motorische System (körperliche Bewegung).

\bigskip\noindent
\emph{Geben Sie zwei Beispiele für Alltagssituationen, in denen das 3-Stufenmodell angewendet wird. Unterteilen Sie die Alltagssituation dabei in die 3 Stufen und geben Sie für jede Stufe die jeweilige Aktion an.}
\par\medskip\noindent
Das 3-Stufenmodell wendet man im Fahrschulunterricht an, um zu modellieren, wie man in einer Gefahrensituation reagiert. Hier entsteht ein meist visueller Reiz, der Wahrgenommen werden muss, der als Gefahrensituation erkannt werden muss, und idealerweise wird als Reaktion eine Gefahrenbremsung eingeleitet. Dieser Prozess dauert ungefähr \SI{100}{\milli\second}.

Beim Schießsport wendet man ebenso das 3-Stufenmodell an. Man zielt mit seiner Waffe auf eine Zielscheibe, und versucht dabei, die Waffe so stetig wie möglich zu halten. Normalerweise hat man bei der Schießübung schon das ein oder andere Bier intus, was dabei hilfreich ist. Ständig kommen visuelle Reize im Auge an. Aber erst wenn die Kimme im Korn ist und der Zielapparat ins Schwarze zielt, wird das wahrgenommen, und es entsteht eine Reaktion, dass der Finger den Abzug durchzieht. 

\subsection*{Aufgabe 2: Wahrnehmung}
\emph{Was ist Vektion und wie kommt Sie zu Stande?}
\par\medskip\noindent
Vektion ist ein Symptom davon, dass die Wahrnehmung verschiedener Sinne nicht isoliert, sondern als Gesamtheit geschieht und interpretiert wird. Wenn man eine bewegte Szenerie sieht, hat man selbst das Gefühl, sich zu bewegen. Das kommt zum Beispiel dann vor, wenn man in einem (stehenden) Zug sitzt, neben dem ein Zug abfährt. Die Bewegung, die man dann sieht, vermittelt einem das Gefühl, dass man sich selbst bewegt. 

\bigskip\noindent
\emph{Nennen Sie den Unterschied zwischen Reiz und Wahrnehmung.}
\par\medskip\noindent
Ein Reiz ist eine physikalische, messbare Einheit und bestimmt, in welchem Ausmaß unsere Sinne erreicht werden. Die Wahrnehmung ist aber nur eine Interpretation dieses Reizes. Diese hängt von dem Individuum ab, und ist nur eine Interpretation der Realität.

\bigskip\noindent
\emph{Woran liegt es, dass sich Wahrnehmungen des gleichen Reizes unterscheiden können? Nennen Sie 2 Gründe.}
\par\medskip\noindent
Reize werden relativ und nicht absolut interpretiert. Also werden die selben Reize in unterschiedlichen Kontexten unterschiedlich wahrgenommen. Beispielsweise wird ein Lichtreiz einer konstanten Intesität als heller empfunden, wenn die Umgebung sehr dunkel ist (nachts), verglichen mit einer hellen Umgebung (tagsüber). Ein anderes Beispiel ist, dass Wahrnehmung zum Beispiel von der Kultur abhängt. So könnte es zum Beispiel passieren, dass ein Christ in einer Wolke einen Jesus sieht, ein Moslem einen Mohammed, und ein Meteorologe kondensierte suspendierte Wassertröpfchen.

\subsection*{Aufgabe 3: Frühe Wahrnehmung}
\emph{Beschreiben Sie kurz in eigenen Worten, was man unter „früher Wahrnehmung“ versteht. Geben Sie außerdem 3 Beispiele für frühe Wahrnehmung an, die nicht in den Vorlesungsfolien genannt werden.}
\par\medskip\noindent
Frühe Wahrnehmung bezeichnet alles, was in \SI{10}{\milli\second} oder schneller wahrgenommen wird. So können zum Beispiel Farbe (starke Farbunterschiede) und Richtungen erkannt werden, bevor das Bild an sich erkannt wird. Wenn man zum Beispiel ein Bild mit 15 Kreisen und einem Stab sieht, dann kann man diesen Stab früh Wahrnehmen. Wenn man ein Bild mit 10 Rechtecken, die alle mit einer Seite waagerecht stehen bis auf eins, was leicht schräg steht, dann kann man dies früh wahrnehmen. 

\subsection*{Aufgabe 4: Depth Cue Theorie}
\emph{Nennen Sie die 3 Kategorien von Depth Cues.}
\par\medskip\noindent
Es gibt binokulare Depth Cues, die dadurch entstehen, dass wir in Stereo (also mit zwei Augen) sehen. Die zweite Kategorie von Depth Cues sind Pictorial Depth Cues. Dies sind alle Depth Cues, die von einem Bild ausgehen (also mit nur einem Auge erkannt werden können). Die dritte Kategorie sind dynamische Depth Cues, die aus der Bewegung entstehen.

\bigskip\noindent
\emph{Geben Sie für jede Kategorie jeweils ein Beispiel an und erklären Sie dieses kurz.}
\par\medskip\noindent
Bei der \emph{Stereoskopie} arbeiten beide Augen zusammen, um Tiefe zu erkennen. Wird ein Gegenstand betrachtet, so sieht, bedingt durch den Abstand der Augen, das linke Auge das Objekt aus einem geringfügig anderen Winkel als das rechte Auge. Im Gehirn werden diese beiden Einzelbilder zu einem dreidimensionalen Bild zusammengefügt.

Bei den \emph{Pictorial Depth Cues} werden gewisse Eigenschaften von Bildern ausgewertet, um Tiefe zu erkennen. Bei der Linearperspektive wird durch Verkleinerung der Objekten auf dem Bild mit zunehmender Entfernung vom Betrachter so dargestellt, dass der Eindruck von dem dreidimensionalen Raum entsteht, obwohl man nur 2D Bild betrachtet.

Bei den \emph{Dynamischen Depth Cues} werden Bewegungen ausgewertet, um Tiefen zu erkennen. Bewegung von 2D-Körpern, die den Eindruck der 3D-Tiefenkomponente auslösen. Wenn diese Körper sich nicht bewegen, kann keine 3D-Anordnung erkannt werden.

\end{document}